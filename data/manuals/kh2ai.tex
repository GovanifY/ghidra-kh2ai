\documentclass[openany,oneside]{memoir}
\usepackage[utf8]{inputenc}
\usepackage[english]{babel}
\usepackage{blindtext}
\usepackage{fancyhdr}
\usepackage[explicit]{titlesec}
\usepackage{ifthen}
\usepackage[bookmarks=true]{hyperref}
\usepackage{bookmark}
\usepackage{xinttools}
\usepackage{xstring}
\usepackage{ifthen}
\usepackage{listings}

\newcounter{bitindex}

% ISA version
\newcommand\version{v0.1}


\pagestyle{fancy}
\fancyhf{}
\fancyhead[R]{KH2Ai ISA \version}
\fancyhead[L]{Gauvain "GovanifY" Roussel-Tarbouriech}
\fancyfoot{-\thepage-}
% alternating footer
\fancyfoot[L]{\ifthenelse{\isodd{\value{page}}}{CC BY 4.0}{}}
\fancyfoot[R]{\ifthenelse{\isodd{\value{page}}}{}{CC BY 4.0}}
\renewcommand{\headrulewidth}{2pt}
\renewcommand{\footrulewidth}{1pt}

\makeatletter
\renewcommand\@seccntformat[1]{}
\makeatother

\titleformat{\chapter}[display]
  {\normalfont\bfseries}{}{0pt}{\Huge}

\newcommand{\Line}{\rule{\linewidth}{1.0mm}}
\newcommand{\Lineless}{\rule{\linewidth}{0.7mm}}
\newcommand{\Chapter}[1]{
\pagestyle{fancy}
\chapter{#1}
\begin{vplace}[0.7]
{\Huge   \null\hfill{\textbf{\thechapter.  #1}}} \\
\Line \\
\end{vplace} 
\newpage
}

\newcommand{\Main}[2]{
\begin{vplace}[0.7]
{\Huge   \null\hfill{\textbf{#1}}} \\
\Line \\
\huge \null\hfill \textbf{#2}
\end{vplace}
\newpage
}

\newcommand{\ISA}[7]{
\StrLen{#4}[\exclen]
\StrLen{#6}[\notelen]

\section{\huge #1}
\Lineless \\
\textbf{Operation Code} \\ \\
\bitpicture {#7} \\ \\
\textbf{Format} \\ 
\hspace*{0.5cm} #2 \\ \\
\textbf{Description} \\ 
\hspace*{0.5cm} #3  \\ \\
\ifthenelse{\equal{\exclen}{0}}{}{
\textbf{Exceptions} \\ 
\hspace*{0.5cm} #4 \\ \\
}
\textbf{Operations} %\\ 
%\hspace*{0.5cm} \lstinputlisting{#5} \\ \\
\lstinputlisting[xleftmargin=0.5cm]{#5} 
%\\ \\
\ifthenelse{\equal{\notelen}{0}}{}{
\textbf{Programming notes} \\ 
\hspace*{0.5cm} #6 \\ \\
}
\newpage
}

\newcommand\bitpicture [1]{%
  \StrLen{#1}[\bitlen]
  \StrLeft{#1}{32}[\bita]
  \StrRight{#1}{32}[\bitb]
  \ifthenelse{\bitlen > 16}{\setlength{\unitlength}{0.9mm}}{\setlength{\unitlength}{1.6mm}}
  \setlength{\fboxsep}{0mm}
  \begin{picture}(130,16)
    % sign bit
  \ifthenelse{\bitlen > 32}{
  \setcounter{bitindex}{1}%
  \xintFor* ##1 in {\bita}
  \do
  {\put(\numexpr 1+4*\value{bitindex},4){\framebox(4,8){##1}}%
   \stepcounter{bitindex}}% 
   
   \setcounter{bitindex}{1}%
  \xintFor* ##1 in {\bitb}
  \do
  {\put(\numexpr 1+4*\value{bitindex},-12){\framebox(4,8){##1}}%
   \stepcounter{bitindex}} 
  }{
  \setcounter{bitindex}{1}% 
  \xintFor* ##1 in {#1}
  \do
  {\put(\numexpr 1+4*\value{bitindex},4){\framebox(4,8){##1}}%
   \stepcounter{bitindex}}%
   }
  % upper labels
  %\put(0,14){\scriptsize{MSB}}
  %\put(126,14){\scriptsize{LSB}}
  %lower labels
 % \put(3,0){\scriptsize{S}}
%  \put(7,0){\line(0,1){2}}
%  \put(7,1){\vector(1,0){8}}
%  \put(16,0){\scriptsize{Exponent}}
%  \put(37,1){\vector(-1,0){8}}
%  \put(37,0){\line(0,1){2}}
%  \put(39,0){\line(0,1){2}}
%  \put(39,1){\vector(1,0){38}}
%  \put(79,0){\scriptsize{Fraction}}
%  \put(130,1){\vector(-1,0){38}}
%  \put(130,0){\line(0,1){2}}
\end{picture}%
\ifthenelse{\bitlen > 32}{\\ \\ \\}{}
}


% other preamble stuff...
\usepackage{etoolbox}
\patchcmd{\chapter}{\thispagestyle{plain}}{\thispagestyle{fancy}}{}{}

\begin{document}

\Main{Kh2Ai ISA}{\version}

Blabla 
It is also worthy to note that some operations that otherwise do the same thing
are given a different mnemonic depending on the context to be easier to write an
assembler. An example of this can be seen in the PUSH.V and PUSH.L operations,
which, while they both push a value to the stack, one of them is 48bits long and
pushes a raw value while the other is 32bits long and does a relocation on the
encoded address before pushing it, making the different naming needed.

\Chapter{Notational Convention}

\section{Instruction Format of Each Instruction} 
The description of each instruction uses the following format. 

\section{Mnemonic}
Page headings show the instruction mnemonic and a brief description of the function, and the MIPS architecture level.
\section{Instruction Encoding} 
This picture illustrates the bit formats of an instruction word. 
\section{Format} 
This section indicates the instruction formats for the assembler. Lower case indicates variables, corresponding to variable fields in the encoding picture. 
\section{Description Section} 
This section describes the instruction function and operation. 
\section{Exception Section} 
This section shows the exceptions that can be caused by the instructions. 
\section{Operation Section}
This section describes the instruction operations in SLEIGH. You can refer to SLEIGH's own documentation for its notational conventions.
\section{Programming Notes Section}
This section shows the supplementary information about programming when using the instruction.

\Chapter{Instruction Set}

\ISA{PUSH: Push}{b}{c}{}{sleigh/push.v.txt}{f}{01000000010010010000111111010000111111}

\ISA{PUSHA: Push and Add}{b}{c}{}{sleigh/push.a.txt}{f}{01000000010010010000111111010000}

\ISA{PUSHAP: Push and Add to Pointer}{b}{c}{}{sleigh/push.ap.txt}{f}{01000000010010010000111111010000}

\ISA{POP: pop}{b}{c}{}{sleigh/popat.txt}{f}{01000000010010010000111111010000}

\ISA{CFTI: Convert Float To Int}{CFTI}{c}{}{sleigh/cfti.txt}{f}{01000000010010010000111111010000}

\ISA{NEG: convert to NEGative signed number}{NEG}{c}{}{sleigh/neg.txt}{f}{01000000010010010000111111010000}

\ISA{INV: INVert an unsigned value}{INV}{c}{}{sleigh/inv.txt}{f}{01000000010010010000111111010000}

\ISA{EQZ: conditional is EQual Zero}{EQZ}{c}{}{sleigh/eqz.txt}{f}{01000000010010010000111111010000}

\ISA{ABS: convert to ABSolute value}{ABS}{c}{}{sleigh/abs.txt}{f}{01000000010010010000111111010000}

\ISA{MSB: return Most Significant Bit}{MSB}{c}{}{sleigh/msb.txt}{f}{01000000010010010000111111010000}

\ISA{INFO: conditional INFerior to One}{INFO}{c}{}{sleigh/info.txt}{f}{01000000010010010000111111010000}

\ISA{NEQZ: conditional Not Equal to Zero}{NEQZ}{c}{}{sleigh/neqz.txt}{f}{01000000010010010000111111010000}

\ISA{MSBI: return Most Significant Bit Inverted}{MSBI}{c}{}{sleigh/msbi.txt}{f}{01000000010010010000111111010000}

\ISA{IPOS: Conditional Is POSitive}{IPOS}{c}{}{sleigh/ipos.txt}{f}{01000000010010010000111111010000}

\ISA{CITF: Convert Int To Float}{CITF}{c}{}{sleigh/citf.txt}{f}{01000000010010010000111111010000}

\ISA{NEGF: convert to NEGative Float}{NEGF}{c}{}{sleigh/negf.txt}{f}{01000000010010010000111111010000}

\ISA{ABSF: convert to ABSolute Float}{ABSF}{c}{}{sleigh/absf.txt}{f}{01000000010010010000111111010000}

\ISA{INFZF: Conditional INFerior to Zero Float}{INFZF}{c}{}{sleigh/infzf.txt}{f}{01000000010010010000111111010000}

\ISA{INFOEZF: Conditional INFerior Or Equal to Zero
Float}{INFOEZF}{c}{}{sleigh/infoezf.txt}{f}{01000000010010010000111111010000}

\ISA{EQZF: conditional is EQual Zero Float}{EQZF}{c}{}{sleigh/eqzf.txt}{f}{01000000010010010000111111010000}

\ISA{NEQZF: conditional Not Equal to Zero Float}{NEQZF}{c}{}{sleigh/neqzf.txt}{f}{01000000010010010000111111010000}

\ISA{SUPOEZF: conditional SUPerior Or Equal to Zero
Float}{SUPOEZF}{c}{}{sleigh/supoezf.txt}{f}{01000000010010010000111111010000}

\ISA{SUPZF: conditional SUPerior to Zero Float}{SUPZF}{c}{}{sleigh/supzf.txt}{f}{01000000010010010000111111010000}

\ISA{ADD: ADDition}{ADD}{Retrieves the last 2 values pushed on to
the stack and applies an addition between them, pushing back the result to the
stack.}{}{sleigh/add.txt}{}{0000011000000000}

\ISA{SUB: SUBstraction}{SUB}{Retrieves the last 2 values pushed on to
the stack and applies a substraction between them, pushing back the result to the
stack.}{}{sleigh/sub.txt}{}{0100011000000000}

\ISA{MUL: MULtiplication}{MUL}{Retrieves the last 2 values pushed on to
the stack and applies a multiplication between them, pushing back the result to the
stack.}{}{sleigh/mul.txt}{}{1000011000000000}

\ISA{DIV: DIVision}{DIV}{Retrieves the last 2 values pushed on to
the stack and applies a division between them, pushing back the result to the
stack.}{}{sleigh/div.txt}{}{1100011000000000}

\ISA{MOD: MODulo}{MOD}{Retrieves the last 2 values pushed on to
the stack and applies a modulo between them, pushing back the result to the
stack.}{}{sleigh/mod.txt}{}{0000011000000001}

\ISA{AND: logical AND}{AND}{Retrieves the last 2 values pushed on to
the stack and applies a logical and between them, pushing back the result to the
stack.}{}{sleigh/and.txt}{}{0100011000000001}

\ISA{OR: logical OR}{OR}{Retrieves the last 2 values pushed on to
the stack and applies a logical or between them, pushing back the result to the
stack.}{}{sleigh/or.txt}{}{1000011000000001}

\ISA{XOR: logical eXclusive OR}{XOR}{Retrieves the last 2 values pushed on to
the stack and applies an exclusive or between them, pushing back the result to the
stack.}{}{sleigh/xor.txt}{}{1100011000000001}

\ISA{SLL: Shift Logical Left}{SLL}{Retrieves the last 2 values pushed on to
the stack and applies a left logical shift between them, pushing back the result to the
stack.}{}{sleigh/sll.txt}{}{0000011000000010}

\ISA{SRA: Shift Right Arithmetic}{SRA}{Retrieves the last 2 values pushed on to
the stack and applies a right arithmetic shift between them, pushing back the result to the
stack.}{}{sleigh/sra.txt}{}{0100011000000010}

\ISA{NEQZV: conditional Not EQual to Zero with stack
Values}{NEQZV}{Retrieves the last 2 values pushed on to the stack and verifies if
both are equal to zero, pushing back the result to the
stack.}{}{sleigh/neqzv.txt}{}{1100011000000010}

\ISA{EQZV: conditional EQual to Zero with stack
Values}{EQZV}{Retrieves the last 2 values pushed on to the stack and verifies if
both are equal to zero, pushing back the result to the
stack.}{}{sleigh/eqzv.txt}{}{1000011000000010}

\ISA{ADDF: ADDition with Float values}{ADDF}{Retrieves the last 2 values pushed on
to the stack and apply an addition onto them, pushing back the result to the
stack.}{}{sleigh/addf.txt}{This function exclusively deals with floating numbers}{0001011000000000}

\ISA{SUBF: SUBstraction with Float values}{SUBF}{Retrieves the last 2 values pushed on
to the stack and apply a substraction onto them, pushing back the result to the
stack.}{}{sleigh/subf.txt}{This function exclusively deals with floating numbers}{0101011000000000}

\ISA{MULF: MULtiplication with Float values}{MULF}{Retrieves the last 2 values pushed on
to the stack and apply a multiplication onto them, pushing back the result to the
stack.}{}{sleigh/mulf.txt}{This function exclusively deals with floating numbers}{1001011000000000}

\ISA{DIVF: DIVision with Float values}{DIVF}{Retrieves the last 2 values pushed on
to the stack and apply a division onto them, pushing back the result to the
stack.}{}{sleigh/divf.txt}{This function exclusively deals with floating numbers}{1101011000000000}

\ISA{MODF: MODulo with Float values}{MODF}{Retrieves the last 2 values pushed on
to the stack and apply a modulo onto them, pushing back the result to the
stack.}{}{sleigh/modf.txt}{This function exclusively deals with floating numbers}{0001011000000001}

\ISA{JMP: JuMP}{JMP ri, addr}{Change the control flow to the given address addr
and saves the instruction following it as the return pointer.}{}{sleigh/jmp.txt}{Argument ri is currently
unknown. The following address relocation formula is applied when decoding a
into addr: $addr=inst\_start+(a*2)+4$ where inst\_start is the beginning of the
instruction.}{ii001000iiiiiiiiaaaaaaaaaaaaaaaa}

\ISA{EXIT: EXIT}{EXIT ri}{Completely stops the execution flow of the AI Parser
with return code ri}{}{sleigh/exit.txt}{In the bitwise encoding ri is encoded as
$r=ri-1$}{000010010000000r}

\ISA{RET: RETurn}{RET}{Stops the execution flow and return back to the last
saved function call}{}{sleigh/ret.txt}{}{1000100100000000}

\ISA{PUSH.CA: PUSH CAched value}{PUSHCA}{c}{}{sleigh/push.ca.txt}{f}{1100100100000000}

\ISA{PUSH.C: PUSH Copy}{PUSHC}{c}{}{sleigh/push.c.txt}{}{0100100100000001}

\ISA{SIN: SINus}{SIN}{Retrieves the latest value pushed to the stack and apply a
sinus onto it, pushing it to the stack}{}{sleigh/sin.txt}{Radians are used as input.
Radians used are modulo $[\pi-2\pi]$}{1000100100000001}

\ISA{COS: COSinus}{COS}{Retrieves the latest value pushed to the stack and apply a
cosinus onto it, pushing it to the stack}{}{sleigh/cos.txt}{Radians are used as input.
Radians used are modulo $[\pi-2\pi]$}{1100100100000001}

\ISA{DEGR: DEGrees to Radians}{DEGR}{Retrieves the last element pushed to the stack
and converts it to radians, pushing it to the stack}{}{sleigh/degr.txt}{Radians used
are modulo $[\pi-2\pi]$}{0000100100000010}

\ISA{RADD: RADians to Degrees}{RADD}{Retrieves the last element pushed to the
stack and converts it to degrees, pushing it to the stack}{}{sleigh/radd.txt}
{Radians used are modulo $[\pi-2\pi]$}{0100100100000010}

\ISA{SYSCALL: SYStem CALL}{b}{c}{}{sleigh/syscall.txt}{f}{TODO}


\Chapter{System Calls}

\section{Introduction}
What is a system call blabla.
None of them are currently documented, they are available at address 0x0034dd00
of SLPM\_666.75,  and there is 738 elements if I'm not mistaken. Either try to
guess their arguments 


\Chapter{Known issues}
As this is very much a work-in-progress project, much of the ISA has yet to
stabilize before getting a stable documentation and some issues still exist. You
will find below some of those.

\section{Notable amount of undocumented instructions}
While the disassemblers knows the size of all instructions and is able to get a
complete unbroken output, some functions are still partly or fully unknown and
as such cannot be assembled yet, nor are they understood by the decompiler. Such
instructions will most likely have "unk" in their name.

\section{syscalls function pointers breaks X-Refs}
Sometimes, syscalls take for arguments function pointers. An analyzer has been
created to be able to analyze this specific case but I have been unable to find
a way to get a similar instruction but resolving the relocation without breaking
the assembler.
As such pointers are written down as comments next to the instruction. You would
have to use those to verify X-Refs until a better solution is found.

\end{document}
