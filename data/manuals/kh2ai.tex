\documentclass[openany,oneside]{memoir}
\usepackage[utf8]{inputenc}
\usepackage[english]{babel}
\usepackage{blindtext}
\usepackage{fancyhdr}
\usepackage[explicit]{titlesec}
\usepackage{ifthen}
\usepackage[bookmarks=true]{hyperref}
\usepackage{bookmark}
\usepackage{xinttools}
\usepackage{xstring}
\usepackage{ifthen}
\usepackage{listings}

\newcounter{bitindex}

% ISA version
\newcommand\version{v0.1}


\pagestyle{fancy}
\fancyhf{}
\fancyhead[R]{KH2AI ISA \version}
\fancyhead[L]{Gauvain "GovanifY" Roussel-Tarbouriech}
\fancyfoot{-\thepage-}
% alternating footer
\fancyfoot[L]{\ifthenelse{\isodd{\value{page}}}{CC BY 4.0}{}}
\fancyfoot[R]{\ifthenelse{\isodd{\value{page}}}{}{CC BY 4.0}}
\renewcommand{\headrulewidth}{2pt}
\renewcommand{\footrulewidth}{1pt}

\makeatletter
\renewcommand\@seccntformat[1]{}
\makeatother

\titleformat{\chapter}[display]
  {\normalfont\bfseries}{}{0pt}{\Huge}

\newcommand{\Line}{\rule{\linewidth}{1.0mm}}
\newcommand{\Lineless}{\rule{\linewidth}{0.7mm}}
\newcommand{\Chapter}[1]{
\pagestyle{fancy}
\chapter{#1}
\begin{vplace}[0.7]
{\Huge   \null\hfill{\textbf{\thechapter.  #1}}} \\
\Line \\
\end{vplace} 
\newpage
}

\newcommand{\Main}[2]{
\begin{vplace}[0.7]
{\Huge   \null\hfill{\textbf{#1}}} \\
\Line \\
\huge \null\hfill \textbf{#2}
\end{vplace}
\newpage
}

\newcommand{\ISA}[7]{
\StrLen{#4}[\exclen]
\StrLen{#6}[\notelen]

\section{\huge #1}
\Lineless \\
\textbf{Operation Code} \\ \\
\bitpicture {#7} \\ \\
\textbf{Format} \\ 
\hspace*{0.5cm} #2 \\ \\
\textbf{Description} \\ 
\hspace*{0.5cm} #3  \\ \\
\ifthenelse{\equal{\exclen}{0}}{}{
\textbf{Exceptions} \\ 
\hspace*{0.5cm} #4 \\ \\
}
\textbf{Operations} %\\ 
%\hspace*{0.5cm} \lstinputlisting{#5} \\ \\
\lstinputlisting[xleftmargin=0.5cm]{#5} 
%\\ \\
\ifthenelse{\equal{\notelen}{0}}{}{
\textbf{Programming notes} \\ 
\hspace*{0.5cm} #6 \\ \\
}
\newpage
}

\newcommand\bitpicture [1]{%
  \StrLen{#1}[\bitlen]
  \StrLeft{#1}{32}[\bita]
  \StrGobbleLeft{#1}{32}[\bitb]
  \ifthenelse{\bitlen > 16}{\setlength{\unitlength}{0.9mm}}{\setlength{\unitlength}{1.6mm}}
  \setlength{\fboxsep}{0mm}
  \begin{picture}(130,16)
    % sign bit
  \ifthenelse{\bitlen > 32}{
  \setcounter{bitindex}{1}%
  \xintFor* ##1 in {\bita}
  \do
  {\put(\numexpr 1+4*\value{bitindex},4){\framebox(4,8){##1}}%
   \stepcounter{bitindex}}% 
   
   \setcounter{bitindex}{1}%
  \xintFor* ##1 in {\bitb}
  \do
  {\put(\numexpr 1+4*\value{bitindex},-12){\framebox(4,8){##1}}%
   \stepcounter{bitindex}} 
  }{
  \setcounter{bitindex}{1}% 
  \xintFor* ##1 in {#1}
  \do
  {\put(\numexpr 1+4*\value{bitindex},4){\framebox(4,8){##1}}%
   \stepcounter{bitindex}}%
   }
  % upper labels
  %\put(0,14){\scriptsize{MSB}}
  %\put(126,14){\scriptsize{LSB}}
  %lower labels
 % \put(3,0){\scriptsize{S}}
%  \put(7,0){\line(0,1){2}}
%  \put(7,1){\vector(1,0){8}}
%  \put(16,0){\scriptsize{Exponent}}
%  \put(37,1){\vector(-1,0){8}}
%  \put(37,0){\line(0,1){2}}
%  \put(39,0){\line(0,1){2}}
%  \put(39,1){\vector(1,0){38}}
%  \put(79,0){\scriptsize{Fraction}}
%  \put(130,1){\vector(-1,0){38}}
%  \put(130,0){\line(0,1){2}}
\end{picture}%
\ifthenelse{\bitlen > 32}{\\ \\ \\}{}
}


% other preamble stuff...
\usepackage{etoolbox}
\patchcmd{\chapter}{\thispagestyle{plain}}{\thispagestyle{fancy}}{}{}

\begin{document}

\Main{KH2AI ISA}{\version}

Kingdom Hearts 2 is a video game developped by Square Enix that also happens to
be a very good game. As Square loves to reinvent the wheel they decided to make
a custom AI assembler like scripting language for this engine, which also
happens to be pretty inconsistent. 
This document will, in its value of a document, document this language as an
Instruction Set Architecture(ISA) with additional information when necessary.

This booklet is separated into parts:
\begin{itemize}
\item The Notational Convention, explaining how every instruction is defined
\item The Instruction Set, defining every operation in this language
\item The System Calls, documenting calls done by the language outside of its own scope
\item Known issues, if any
\item An appendix for additional documents that might help comprehension
\end{itemize}
It is also worthy to note that some operations that otherwise do the same thing
are given a different mnemonic depending on the context to be easier to write an
assembler. An example of this can be seen in the PUSH.V and PUSH.L operations,
which, while they both push a value to the stack, one of them is 48bits long and
pushes a raw value while the other is 32bits long and does a relocation on the
encoded address before pushing it, making the different naming needed.

\Chapter{Notational Convention}

\section{Instruction Format of Each Instruction} 
The description of each instruction uses the following format:

\section{Mnemonic}
Page headings show the instruction mnemonic and a brief description of the function, and the MIPS architecture level.
\section{Instruction Encoding} 
This picture illustrates the bit formats of an instruction word. 
\section{Format} 
This section indicates the instruction formats for the assembler. Lower case indicates variables, corresponding to variable fields in the encoding picture. 
\section{Description Section} 
This section describes the instruction function and operation. 
\section{Exception Section} 
This section shows the exceptions that can be caused by the instructions. 
\section{Operation Section}
This section describes the instruction operations in SLEIGH. You can refer to
SLEIGH's own documentation for its notational conventions and refer to the
Appendix for the custom SLEIGH notational conventions defined.
\section{Programming Notes Section}
This section shows the supplementary information about programming when using the instruction.


\Chapter{Instruction Set}

\ISA{PUSH.V: PUSH a Value}{PUSH.V
ri}{Pushes a value to the stack.}{}{sleigh/push.v.txt}{}{0000000000000000iiiiiiiiiiiiiiiiiiiiiiiiiiiiiiii}

\ISA{PUSH..L: PUSH a given relocated Label}{PUSH.L la}{Pushes a relocated
label(address pointer) to the stack.}{}{sleigh/push.l.txt}{The relocation
formula is $0x10+(l>>1)$}{1110000000000000llllllllllllllll}

\ISA{PUSH.A: PUSH and Add}{PUSH.A rn, ri}{Pushes to the stack the value
(rn+ri).}{}{sleigh/push.a.txt}{}{rr11000000000000iiiiiiiiiiiiiiii}

\ISA{PUSH.AP: PUSH and Add to Pointer}{PUSH.AP rn, ri}{Pushed to the stack a
pointer toward
(rn+ri).}{}{sleigh/push.ap.txt}{}{rr10000000000000iiiiiiiiiiiiiiii}

\ISA{POP.A: POP and Add}{POP.A rn, ri}{Pops the last value from the stack to the
address (rn+ri).}{}{sleigh/pop.a.txt}{}{rr00000100000000iiiiiiiiiiiiiiii}

\ISA{POP.L: POP at a given relocated Label}{POP.L la}{Pops the latest value from
the stack and stores it at the relocated
label(address pointer) la.}{}{sleigh/push.l.txt}{The relocation
formula is $0x10+(l>>1)$. This opcode is never used in practice as the only way
to use this opcode is to modify the AI's own ram region, which would create
self-modifying code.}{1100000100000000llllllllllllllll}

\ISA{CFTI: Convert Float To Int}{CFTI}{Retrieves the last value pushed on to
the stack and converts it from a signed integer to a floating point value, pushing back the result to the
stack.}{}{sleigh/cfti.txt}{}{0000010100000000}

\ISA{NEG: convert to NEGative signed number}{NEG}{Retrieves the last value pushed on to
the stack and converts it to a negative number, pushing back the result to the
stack.}{}{sleigh/neg.txt}{}{1000010100000000}

\ISA{INV: INVert an unsigned value}{INV}{Retrieves the last value pushed on to
the stack and inverts it, pushing back the result to the
stack.}{}{sleigh/inv.txt}{}{1100010100000000}

\ISA{EQZ: conditional is EQual Zero}{EQZ}{Retrieves the last value pushed on to
the stack and compares it to zero, pushing back the result to the
stack.}{}{sleigh/eqz.txt}{}{0000010100000001}

\ISA{ABS: convert to ABSolute value}{ABS}{Retrieves the last value pushed on to
the stack and converts it to an absolute value, pushing back the result to the
stack.}{}{sleigh/abs.txt}{}{0100010100000001}

\ISA{MSB: return Most Significant Bit}{MSB}{Retrieves the last value pushed on to
the stack and gets back its most significant bit, pushing back the result to the
stack.}{}{sleigh/msb.txt}{}{1000010100000001}

\ISA{INFO: conditional INFerior to One}{INFO}{Retrieves the last value pushed on to
the stack and compares it to one, pushing back the result to the
stack.}{}{sleigh/info.txt}{}{1100010100000001}

\ISA{NEQZ: conditional Not Equal to Zero}{NEQZ}{Retrieves the last value pushed on to
the stack and compares it to zero, pushing back the result to the
stack.}{}{sleigh/neqz.txt}{}{0100010100000010}

\ISA{MSBI: return Most Significant Bit Inverted}{MSBI}{Retrieves the last value pushed on to
the stack and gets back its most significant bit and inverts it, pushing back the result to the
stack.}{}{sleigh/msbi.txt}{}{1000010100000010}

\ISA{IPOS: Conditional Is POSitive}{IPOS}{Retrieves the last value pushed on to
the stack and compares it to zero, pushing back the result to the
stack.}{}{sleigh/ipos.txt}{}{1100010100000010}

\ISA{CITF: Convert Int To Float}{CITF}{Retrieves the last value pushed on to
the stack and converts it from a signed integer to a floating point value, pushing back the result to the
stack.}{}{sleigh/citf.txt}{}{0001010100000000}

\ISA{NEGF: convert to NEGative Float}{NEGF}{Retrieves the last value pushed on to
the stack and converts it to a negative value, pushing back the result to the
stack.}{}{sleigh/negf.txt}{This function exclusively deals with floating numbers}{1001010100000000}

\ISA{ABSF: convert to ABSolute Float}{ABSF}{Retrieves the last value pushed on to
the stack and converts it to an absolute value, pushing back the result to the
stack.}{}{sleigh/absf.txt}{This function exclusively deals with floating numbers}{0101010100000001}

\ISA{INFZF: Conditional INFerior to Zero Float}{INFZF}{Retrieves the last value pushed on to
the stack and compares it to zero, pushing back the result to the
stack.}{}{sleigh/infzf.txt}{This function exclusively deals with floating numbers}{1001010100000001}

\ISA{INFOEZF: Conditional INFerior Or Equal to Zero
Float}{INFOEZF}{Retrieves the last value pushed on to
the stack and compares it to zero, pushing back the result to the
stack.}{}{sleigh/infoezf.txt}{This function exclusively deals with floating numbers}{1101010100000001}

\ISA{EQZF: conditional is EQual Zero Float}{EQZF}{Retrieves the last value pushed on to
the stack and compares it to zero, pushing back the result to the
stack.}{}{sleigh/eqzf.txt}{This function exclusively deals with floating numbers}{0001010100000010}

\ISA{NEQZF: conditional Not Equal to Zero Float}{NEQZF}{Retrieves the last value pushed on to
the stack and compares it to zero, pushing back the result to the
stack.}{}{sleigh/neqzf.txt}{This function exclusively deals with floating numbers}{0101010100000010}

\ISA{SUPOEZF: conditional SUPerior Or Equal to Zero
Float}{SUPOEZF}{Retrieves the last value pushed on to
the stack and compares it to zero, pushing back the result to the
stack.}{}{sleigh/supoezf.txt}{This function exclusively deals with floating numbers}{1001010100000010}

\ISA{SUPZF: conditional SUPerior to Zero Float}{SUPZF}{Retrieves the last value pushed on to
the stack and compares it to zero, pushing back the result to the
stack.}{}{sleigh/supzf.txt}{This function exclusively deals with floating numbers}{1101010100000010}

\ISA{ADD: ADDition}{ADD}{Retrieves the last 2 values pushed on to
the stack and applies an addition between them, pushing back the result to the
stack.}{}{sleigh/add.txt}{}{0000011000000000}

\ISA{SUB: SUBstraction}{SUB}{Retrieves the last 2 values pushed on to
the stack and applies a substraction between them, pushing back the result to the
stack.}{}{sleigh/sub.txt}{}{0100011000000000}

\ISA{MUL: MULtiplication}{MUL}{Retrieves the last 2 values pushed on to
the stack and applies a multiplication between them, pushing back the result to the
stack.}{}{sleigh/mul.txt}{}{1000011000000000}

\ISA{DIV: DIVision}{DIV}{Retrieves the last 2 values pushed on to
the stack and applies a division between them, pushing back the result to the
stack.}{}{sleigh/div.txt}{}{1100011000000000}

\ISA{MOD: MODulo}{MOD}{Retrieves the last 2 values pushed on to
the stack and applies a modulo between them, pushing back the result to the
stack.}{}{sleigh/mod.txt}{}{0000011000000001}

\ISA{AND: logical AND}{AND}{Retrieves the last 2 values pushed on to
the stack and applies a logical and between them, pushing back the result to the
stack.}{}{sleigh/and.txt}{}{0100011000000001}

\ISA{OR: logical OR}{OR}{Retrieves the last 2 values pushed on to
the stack and applies a logical or between them, pushing back the result to the
stack.}{}{sleigh/or.txt}{}{1000011000000001}

\ISA{XOR: logical eXclusive OR}{XOR}{Retrieves the last 2 values pushed on to
the stack and applies an exclusive or between them, pushing back the result to the
stack.}{}{sleigh/xor.txt}{}{1100011000000001}

\ISA{SLL: Shift Logical Left}{SLL}{Retrieves the last 2 values pushed on to
the stack and applies a left logical shift between them, pushing back the result to the
stack.}{}{sleigh/sll.txt}{}{0000011000000010}

\ISA{SRA: Shift Right Arithmetic}{SRA}{Retrieves the last 2 values pushed on to
the stack and applies a right arithmetic shift between them, pushing back the result to the
stack.}{}{sleigh/sra.txt}{}{0100011000000010}

\ISA{NEQZV: conditional Not EQual to Zero with stack
Values}{NEQZV}{Retrieves the last 2 values pushed on to the stack and verifies if
both are equal to zero, pushing back the result to the
stack.}{}{sleigh/neqzv.txt}{}{1100011000000010}

\ISA{EQZV: conditional EQual to Zero with stack
Values}{EQZV}{Retrieves the last 2 values pushed on to the stack and verifies if
both are equal to zero, pushing back the result to the
stack.}{}{sleigh/eqzv.txt}{}{1000011000000010}

\ISA{ADDF: ADDition with Float values}{ADDF}{Retrieves the last 2 values pushed on
to the stack and apply an addition onto them, pushing back the result to the
stack.}{}{sleigh/addf.txt}{This function exclusively deals with floating numbers}{0001011000000000}

\ISA{SUBF: SUBstraction with Float values}{SUBF}{Retrieves the last 2 values pushed on
to the stack and apply a substraction onto them, pushing back the result to the
stack.}{}{sleigh/subf.txt}{This function exclusively deals with floating numbers}{0101011000000000}

\ISA{MULF: MULtiplication with Float values}{MULF}{Retrieves the last 2 values pushed on
to the stack and apply a multiplication onto them, pushing back the result to the
stack.}{}{sleigh/mulf.txt}{This function exclusively deals with floating numbers}{1001011000000000}

\ISA{DIVF: DIVision with Float values}{DIVF}{Retrieves the last 2 values pushed on
to the stack and apply a division onto them, pushing back the result to the
stack.}{}{sleigh/divf.txt}{This function exclusively deals with floating numbers}{1101011000000000}

\ISA{MODF: MODulo with Float values}{MODF}{Retrieves the last 2 values pushed on
to the stack and apply a modulo onto them, pushing back the result to the
stack.}{}{sleigh/modf.txt}{This function exclusively deals with floating numbers}{0001011000000001}

\ISA{JMP: JuMP}{JMP ri, addr}{Change the control flow to the given address addr
and saves the instruction following it as the return pointer.}{}{sleigh/jmp.txt}{Argument ri is currently
unknown. The following address relocation formula is applied when decoding a
into addr: $addr=inst\_start+(a*2)+4$ where inst\_start is the beginning of the
instruction.}{ii001000iiiiiiiiaaaaaaaaaaaaaaaa}

\ISA{EXIT: EXIT}{EXIT ri}{Completely stops the execution flow of the AI Parser
with return code ri}{}{sleigh/exit.txt}{In the bitwise encoding ri is encoded as
$r=ri-1$}{00001001ii000000}

\ISA{RET: RETurn}{RET}{Stops the execution flow and return back to the last
saved function call}{}{sleigh/ret.txt}{}{1000100100000000}

\ISA{PUSH.CA: PUSH CAched value}{PUSH.CA}{Pushes the last cached stack value to
the stack}{}{sleigh/push.ca.txt}{This seems to have the same effect as PUSH.C
but without doing a POP. I have no clue why both of those instructions exist
alongisde another.}{1100100100000000}

\ISA{PUSH.C: PUSH Copy}{PUSH.C}{Pops the latest value from the stack and pushes
it back twice}{}{sleigh/push.c.txt}{}{0100100100000001}

\ISA{SIN: SINus}{SIN}{Retrieves the latest value pushed to the stack and apply a
sinus onto it, pushing it to the stack}{}{sleigh/sin.txt}{Radians are used as input.
Radians used are modulo $[\pi-2\pi]$}{1000100100000001}

\ISA{COS: COSinus}{COS}{Retrieves the latest value pushed to the stack and apply a
cosinus onto it, pushing it to the stack}{}{sleigh/cos.txt}{Radians are used as input.
Radians used are modulo $[\pi-2\pi]$}{1100100100000001}

\ISA{DEGR: DEGrees to Radians}{DEGR}{Retrieves the last element pushed to the stack
and converts it to radians, pushing it to the stack}{}{sleigh/degr.txt}{Radians used
are modulo $[\pi-2\pi]$}{0000100100000010}

\ISA{RADD: RADians to Degrees}{RADD}{Retrieves the last element pushed to the
stack and converts it to degrees, pushing it to the stack}{}{sleigh/radd.txt}
{Radians used are modulo $[\pi-2\pi]$}{0100100100000010}

\ISA{SYSCALL: SYStem CALL}{syscall ri, ra}{Executes a System Call, using the
stack as arguments}{}{sleigh/syscall.txt}{Refer to the syscall own documentation
chapter for more information about this
instruction.}{ii00101000000000aaaaaaaaaaaaaaaa}



\Chapter{System Calls}

\section{Introduction}
KH2AI has an instruction used to call some functions into the base game, which
we call syscall, short for System Call. 
None of them are currently documented, they are available at address 0x0034dd00
of SLPM\_666.75, which is Kingdom Hearts 2 Final Mix ELF file. If you want to
contribute you can submit your syscall findings at
\url{https://framaforms.org/kh2ai-report-errata-1577102965} for them to be
incorporated into the next release of the ISA.


\Chapter{Known issues}
As this is very much a work-in-progress project, much of the ISA has yet to
stabilize before getting a stable documentation and some issues still exist. You
will find below some of those.

\section{Notable amount of undocumented instructions}
While the disassemblers knows the size of all instructions and is able to get a
complete unbroken output, some functions are still partly or fully unknown and
as such cannot be assembled yet, nor are they understood by the decompiler. Such
instructions will most likely have "unk" in their name.

\section{syscalls function pointers breaks X-Refs}
Sometimes, syscalls take for arguments function pointers. An analyzer has been
created to be able to analyze this specific case but I have been unable to find
a way to get a similar instruction but resolving the relocation without breaking
the assembler.
As such pointers are written down as comments next to the instruction. You would
have to use those to verify X-Refs until a better solution is found.


\Chapter{Appendix}
\section{SLEIGH additional notational convention}
\lstinputlisting[breaklines=true]{../languages/base.sinc} 

\end{document}
